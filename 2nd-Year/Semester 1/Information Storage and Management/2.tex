\documentclass[12pt]{article}
\usepackage{listings}
\usepackage{color}

\definecolor{dkgreen}{rgb}{0,0.6,0}
\definecolor{gray}{rgb}{0.5,0.5,0.5}
\definecolor{mauve}{rgb}{0.58,0,0.82}

\lstset{frame=tb,
  language=SQL,
  aboveskip=3mm,
  belowskip=3mm,
  showstringspaces=false,
  columns=flexible,
  basicstyle={\small\ttfamily},
  numbers=none,
  numberstyle=\tiny\color{gray},
  keywordstyle=\color{blue},
  commentstyle=\color{dkgreen},
  stringstyle=\color{mauve},
  breaklines=true,
  breakatwhitespace=true,
  tabsize=3
}
\begin{document}
{\centering
\section*{Information Storage and Management}
\section*{Lecture 2}
\indent\today
}

\subsection*{Labs}
Will commence next week on Thursdays: G.24 and G.21

\subsection*{Tables}
A typcial DB has:
\begin{itemize}
    \item Name
    \item Columns (With types)
    \item Attributes
\end{itemize}
\subsection*{Primary Key}
A \textbf{Primary Key} must be unique, constain values and cannot be NULL
\newline
\newline
Usage of primary keys:
\begin{lstlisting}
    CREATE TABLE Customers (
        ID INT NOT NULL, 
        NAME VARCHAR (20) NOT NULL, 
        AGE INT NOT NULL, 
        ADDRESS CHAR (25) , 
        SALARY DECIMAL (18, 2), 
        PRIMARY KEY (ID, NAME) -- See multiple attributes in prim key 
    )
\end{lstlisting}
\subsection*{Foreign Key}
A \textbf{Foreign Key} is a way of linking a table to another using the other table's Primary Key. Example:
\begin{lstlisting}
    CREATE TABLE Orders (
        OrderID int NOT NULL,
        OrderNumber int NOT NULL,
        PersonID int,
        PRIMARY KEY (OrderID),
        FOREIGN KEY (PersonID) REFERENCES Persons(PersonID)
    ); 
\end{lstlisting}
\subsection*{Data Types}
\begin{itemize}
    \item Numbers: \texttt{INT, LONGINT, TINYINT, NUMERIC, FLOAT, DOUBLE} \newline
        \textit{Int is by default implemented to be 32 bit. Use it by default for ints to reduce memory usage}
    \item \texttt{NOT NULL} vs \texttt{NULL}: whether to allow empty value.
    \item Strings: \texttt{VARCHAR} etc.
    \item Others include Dates, etc
\end{itemize}
\subsection*{Aggregation} 
\begin{itemize}
    \item \texttt{AVG}: 
        \begin{lstlisting}
    SELECT AVG (salary) FROM instructor WHERE deptname= `Comp. Sci.'
        \end{lstlisting}
    \item Grouping: 
        \begin{lstlisting}
    SELECT ID
    FROM instructor
    GROUP BY deptname -- Will make the select query inside this group
        \end{lstlisting}
\end{itemize}
\subsection*{Modifying tables}
\begin{lstlisting}
    DROP TABLE whatever -- Deletes
    ALTER TABLE whatever ADD COLUMN idss
\end{lstlisting}
\end{document}
