\documentclass[12pt]{article}
\usepackage{amsmath}
\begin{document}
{\centering
\section*{Computer Architecture}
\section*{Lecture 1}
\indent\today
}

\subsection*{Abstraction}
\begin{itemize}
    \item You'd interface with a `black box' by using a `knob of sorts' 
\end{itemize}
\subsection*{Performance}
Performance is affected by:
\begin{itemize}
    \item Response time: \textit{The execution of a specific task}
    \item Throughput
\end{itemize}
\textrm{Power of a processor can be calculated using: \newline}
Power = Capacitive Power * Voltage^2 \newline
\begin{itemize}
    \item Elapsed time
        \begin{itemize}
            \item Total to complete task
            \item Determines System Performance
        \end{itemize}
    \item CPU time: Time spent ot do each job
\end{itemize}

\subsection*{CPU Clocking}
\begin{itemize}
    \item CPU exec time for program:
    \begin{equation}
        Cpu time = Clock cycles \times Cycle time = \frac{CPU clock cycles}{Clock rate}
    \end{equation}
\end{itemize}
\subsection*{Instruction Count \& Cycles Per Instruction (\textit{CPI})}
\begin{equation}
    ClockCycles = InstructCount \times CyclesPerInstruction
\end{equation}
\begin{equation}
    CPUTime=InstructCount \times CPI \times ClockCycleTime
\end{equation}
\begin{equation}
    CPUTime = \frac{InstructCount \times CPI}{Clock Rate}
\end{equation}

\end{document}
